\documentclass{article}

\usepackage[utf8]{vietnam}
\usepackage[a4paper, left=2cm, right=2cm, top=2cm, bottom=2cm]{geometry}
\usepackage{scrextend}
\usepackage{amsmath, nccmath}
\usepackage{geometry}
\usepackage{setspace}
\usepackage{graphics}
\usepackage{amsmath, amsthm, amssymb, amsfonts}
\usepackage{fancyhdr}

\pagestyle{fancy}
\fancyhf{}
\rhead{Nguyễn Minh Đăng-20230022}
\fancyfoot[C]{\thepage}
\onehalfspacing
\pagenumbering{arabic}
\onehalfspacing
\changefontsizes{13pt}

\begin{document}

\date{\today}

\textwidth=450pt\oddsidemargin=0pt
\begin{titlepage}
\begin{center}
{{\large{\textsc{TRƯỜNG ĐẠI HỌC KHOA HỌC TỰ NHIÊN \\ ĐẠI HỌC QUỐC GIA TP. HỒ CHÍ MINH}}}} \rule[0.1cm]{15.8cm}{0.1mm}
\rule[0.5cm]{15.8cm}{0.6mm}

\vspace{2cm}

{\Large{\bf THỰC TẬP CHUYÊN ĐỀ 1 }}
\end{center}
\vspace{15mm}
\begin{center}
{\LARGE{\bf BÀI 6: THỜI GIAN CHẾT VÀ HIỆU SUẤT GHI}}
\end{center}
\vspace{4cm}
\par
\noindent
\begin{minipage}[h]{\linewidth}
{\large{\bf Họ và tên - Mã số sinh viên:\\
Nguyễn Minh Đăng - 20230022\\}}
\end{minipage}

\vspace{8cm}

\begin{center}
{\large{\bf TP. HỒ CHÍ MINH \\
28/03/2023.}}
\end{center}
\end{titlepage}

\newpage
\clearpage\thispagestyle{empty}\addtocounter{page}{-1} 
\clearpage
\mbox{}
 % creates a blank space to fill the page
\newpage

\section*{\centering Báo Cáo Kết Quả Thực Nghiệm}
\vspace{1cm}

\setcounter{section}{1}
\subsection{Dụng cụ}
\begin{itemize}
	\item Loại detector sử dụng: Geiger-Müller
	\item Cao thế: 900 (Volt), thời gian đo: 60 giây
	\item Kích thước tinh thể nhấp nháy: 3,31 (cm)
	\item Loại nguồn sử dụng
		\begin{enumerate}
			\item Nguồn yếu: Co-60, hoạt độ: 1 $\mu$Ci
			\item Nguồn mạnh: Eu-152, hoạt độ: 1 $\mu$Ci
		\end{enumerate}
	\item Khoảng cách giữa detector và nguồn: 4,2 (cm)
\end{itemize}
\subsection{Bảng số liệu}
Bảng số liệu số đếm (số đếm/s)
\begin{table}[!ht]
	\centering
	\begin{tabular}{cccccccc}
		\hline
		\textbf{STT} & $\mathbf{m_p}$ & $\mathbf{m_1}$ & $\mathbf{m_2}$ & $\mathbf{m_{12}}$ & $\mathbf{n_1}$ & $\mathbf{n_2}$ & $\mathbf{n_{12}}$ \\ \hline
		\textbf{1} & 0,5167 & 2,3333 & 6,2000 & 9,1833 & 2,3771 & 6,5191 & 9,9013 \\
		\textbf{2} & 0,6667 & 2,6333 & 7,2333 & 8,6833 & 2,6893 & 7,6715 & 9,3225 \\
		\textbf{3} & 0,6833 & 2,5167 & 7,2000 & 8,5833 & 2,5677 & 7,6340 & 9,2074 \\
		\textbf{4} & 0,4000 & 2,3833 & 8,5833 & 8,6167 & 2,4290 & 9,2074 & 9,2457 \\
		\textbf{5} & 0,7000 & 2,7833 & 7,2333 & 9,2167 & 2,8459 & 7,6715 & 9,9401 \\
		\textbf{6} & 0,4167 & 2,2167 & 7,2000 & 8,5500 & 2,2562 & 7,6340 & 9,1690 \\
		\textbf{7} & 0,5167 & 2,2333 & 7,7333 & 8,5333 & 2,2734 & 8,2363 & 9,1498 \\
		\textbf{8} & 0,5000 & 2,3500 & 6,7667 & 9,1833 & 2,3944 & 7,1486 & 9,9013 \\
		\textbf{9} & 0,5500 & 2,2333 & 6,5167 & 8,8500 & 2,2734 & 6,8702 & 9,5149 \\
		\textbf{10} & 0,4667 & 2,5667 & 6,5167 & 8,7167 & 2,6198 & 6,8702 & 9,3610 \\ \hline
		\textbf{Trung bình} & 0,5417 & 2,4250 & 7,1183 & 8,8117 & 2,4726 & 7,5463 & 9,4713 \\
		\textbf{Sai số} & 0,7360 & 1,5572 & 2,6680 & 2,9684 & 1,5725 & 2,7470 & 3,0775 \\ \hline
	\end{tabular}
\end{table}

\subsection{Thời gian chết}

Ta có thời gian chết
\begin{align}
	\tau = \frac{x\Big(1- \sqrt{1-z}\Big)}{y}
\end{align}
\\
Trong đó:
\begin{fleqn}[\parindent]
\begin{equation*}
\begin{split}
& x = m_1m_2-m_pm_{12}  \\
& y = m_1m_2(m_{12}+m_p)-m_pm_{12}(m_1+m_2) \\
& z = \frac{y(m_1+m_2-m_{12}-m_p)}{x^2} 
\end{split}
\end{equation*}
\end{fleqn}
\\
Áp dụng, ta có:
\begin{align*}
& x = 2,4250\times 7,1183 - 0,5417\times 8,8117 = 12,4886  \\
& y = 2,4250\times 7,1183(8,8117 + 0,5417)-0,5417\times 8,8117(2,4250 + 7,1183) \\
& \ \text{   }= 115,9042 \\
& z = \frac{115,9042(2,4250+7,1183-8,8117-0,5417)}{{12,4886}^2} = 0,1411 
\end{align*}
Thế vào công thức (1), ta có:
\begin{align*}
	\tau = \frac{12,4886(1-\sqrt{1-0,1411})}{115,9042} = 0,0079 \ (s)
\end{align*}

\subsection{Hiệu suất ghi}

Ta có công thức tính hiệu suất ghi
\begin{align}
	F_e = \frac{n}{A\times G \times \upsilon}
\end{align}
\\
Trong đó:
\begin{fleqn}[\parindent]
\begin{equation*}
\begin{split}
& n = \bar{n}_1: \text{ Tốc độ đếm nguồn Co-60}  \\
& A: \text{ Cường độ nguồn} \\
& G: \text{ Hệ số hình học} \\
& \upsilon : \text{ số lượng tử suất hiện trong mỗi phân rã} 
\end{split}
\end{equation*}
\end{fleqn}
\\
Ta có hệ số hình học $G$:
\begin{align*}
	G = \frac{1 - \frac{a}{\sqrt{a^2 + r^2}}}{2} = \frac{1 - \frac{4,2}{\sqrt{{4,2}^2 + (3,31/2)^2}}}{2} = 0,0348
\end{align*}
Với: $n=2,4726 \text{ (số đếm/s)};\ A = 3,7\times 10^4 \text{ (pr/s)};\ \upsilon = 2$
\\
Từ đây ta có hiệu suất ghi ở công thức (2)
\begin{align*}
	F_e = \frac{2,4726}{3,7\times 10^4 \times 0,0348 \times 2} \times 100\% = 0,0960 \ (\%)
\end{align*}

\newpage
\clearpage\thispagestyle{empty}\addtocounter{page}{-1} 
\clearpage
\mbox{}
 % creates a blank space to fill the page
\newpage
\end{document}